\documentclass[12pt,a4paper]{article}
\usepackage[utf8]{inputenc}
\usepackage[T1]{fontenc}
\usepackage{amsmath}
\usepackage{amsfonts}
\usepackage{amssymb}
\usepackage{graphicx}
\usepackage{listings}
\usepackage{color}
\usepackage{tikz}
\usepackage{float}
\usepackage{hyperref}

% Define colors for code listings
\definecolor{codegreen}{rgb}{0,0.6,0}
\definecolor{codegray}{rgb}{0.5,0.5,0.5}
\definecolor{codepurple}{rgb}{0.58,0,0.82}
\definecolor{backcolour}{rgb}{0.95,0.95,0.92}

% Listing style for MATLAB code
\lstdefinestyle{matlab}{
    backgroundcolor=\color{backcolour},   
    commentstyle=\color{codegreen},
    keywordstyle=\color{magenta},
    numberstyle=\tiny\color{codegray},
    stringstyle=\color{codepurple},
    basicstyle=\ttfamily\footnotesize,
    breakatwhitespace=false,         
    breaklines=true,                 
    captionpos=b,                    
    keepspaces=true,                 
    numbers=left,                    
    numbersep=5pt,                  
    showspaces=false,                
    showstringspaces=false,
    showtabs=false,                  
    tabsize=2,
    language=MATLAB
}

\title{Wedge Impedance Analysis Program Documentation}
\author{Djamel Ouais and Hussain Alqahtani}
\date{\today}

\begin{document}

\maketitle

\tableofcontents

\section{Overview}
This documentation describes a MATLAB-based program designed for calculating acoustic impedance and wave fields around a wedge structure. The program implements various mathematical functions to compute scattered and direct wave fields using complex acoustic calculations.

\section{Key Components}

\subsection{Main Control Script (testscipt.m)}
The main script initializes and controls the program execution with the following components:
\begin{itemize}
    \item Global parameters initialization including:
    \begin{itemize}
        \item Geometric parameters ($r$, $\phi$, $r'$, $\phi'$)
        \item Wave parameters ($\theta_n$, $\theta_0$)
        \item Physical constants ($c = 340$ m/s for speed of sound)
    \end{itemize}
    \item Frequency range setup (20 Hz to 10 kHz)
    \item Wave field calculations and normalization
    \item Results visualization
\end{itemize}

\subsection{Wave Field Components}
The program consists of several specialized functions for different aspects of the wave field calculations:

\subsubsection{Direct Field Calculations}
\begin{itemize}
    \item \texttt{u\_d.m}: Direct wave field calculator
    \item \texttt{u\_ss.m}: Source-source interaction computation
    \item \texttt{u\_sd.m}: Source-diffraction interaction handler
    \item \texttt{u\_ds.m}: Diffraction-source interaction computation
    \item \texttt{u\_dsw.m}: Diffraction-source-wedge interaction calculator
\end{itemize}

\subsubsection{Mathematical Support Functions}
\begin{itemize}
    \item \texttt{A\_n.m}, \texttt{M\_n.m}: Coefficient calculations
    \item \texttt{P\_l\_m.m}: Legendre polynomial implementations
    \item \texttt{omega\_n.m}: Angular frequency calculations
    \item \texttt{epsy\_n.m}, \texttt{epsy\_cap.m}: Phase calculations
    \item \texttt{g\_small.m}, \texttt{h\_small.m}: Field calculation helper functions
\end{itemize}

\section{Program Flow}
The program follows a systematic approach to compute acoustic fields:

\subsection{Initialization Phase}
\begin{enumerate}
    \item Global parameter setup
    \item Geometric configuration definition
    \item Material properties initialization
\end{enumerate}

\subsection{Computation Phase}
\begin{enumerate}
    \item Frequency range iteration (20 Hz - 10 kHz)
    \item Wave number calculation per frequency
    \item Field computations:
    \begin{itemize}
        \item Normal incidence field
        \item Direct field
        \item Scattered field components
        \item Field combination
    \end{itemize}
\end{enumerate}

\subsection{Output Phase}
\begin{enumerate}
    \item Results normalization
    \item Frequency response plotting
\end{enumerate}

\section{Mathematical Foundation}
The program implements complex acoustic theory including:

\subsection{Wave Propagation}
The wave equation in cylindrical coordinates:
\begin{equation}
    \nabla^2\Phi + k^2\Phi = 0
\end{equation}
where $k$ is the wave number and $\Phi$ is the velocity potential.

\subsection{Scattered Field}
The total field is composed of incident and scattered components:
\begin{equation}
    \Phi_{\text{total}} = \Phi_{\text{incident}} + \Phi_{\text{scattered}}
\end{equation}

\subsection{Impedance Boundary Conditions}
At the wedge surface:
\begin{equation}
    \frac{\partial \Phi}{\partial n} + \beta\Phi = 0
\end{equation}
where $\beta$ is the surface admittance and $n$ is the normal direction.

\section{Usage Instructions}
To use the program:

\begin{enumerate}
    \item Ensure all MATLAB files are in the same directory
    \item Execute \texttt{testscipt.m}
    \item Review the generated plots showing field ratio vs. frequency
\end{enumerate}

\section{Code Structure}
Example of the main calculation loop:

\begin{lstlisting}[style=matlab]
for index = 1:length(f)
    k = 2*pi*f(index)/c;
    normal_quin(index) = exp(-1j*k*r_)/sqrt(k*r_);
    A = u_sw_t();
    if ((isnan(real(A))||(isnan(imag(A))))
        A = 0;
    end
    total_feild(index) = u_d() + A;
end
\end{lstlisting}

\section{Conclusion}
This implementation provides a comprehensive solution for analyzing acoustic behavior around wedge-shaped structures. It combines theoretical acoustic models with numerical methods to provide accurate simulations of wave propagation and interaction phenomena.

\end{document}
